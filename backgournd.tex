	Natarajan \cite{bac1} proposed brain tumor detection method for MRI brain
	images. The MRI brain images are first preprocessed using median filter,
	then segmentation of image is done using threshold segmentation and
	morphological operations are applied and then finally, the tumor region is
	obtained using image subtraction technique. This approach gives the exact
	shape of tumor in MRI brain image. Joshi \cite{bac2} proposed brain tumor
	detection and classification system in MR images by first extracting the
	tumor portion from brain image, then extracting the texture features of the
	detected tumor using Gray Level Co-occurrence Matrix (GLCM) and then
	classified using neuro-fuzzy classifier. Amin and Mageed \cite{bac3}
	proposed neural network and segmentation base system to automatically detect
	the tumor in brain MRI images. The Principal Component Analysis (PCA) is
	used for feature extraction and then Multi-Layer Perceptron (MLP) is used
	classify the extracted features of MRI brain image. The average recognition
	rate is 88.2\% and peak recognition rate is 96.7\%. Sapra \cite{bac4}
	proposed image segmentation technique to detect brain tumor from MRI images
	and then Probabilistic Neural Network (PNN) is used for automated brain
	tumor classification in MRI scans. PNN system proposed handle the process of
	brain tumor classification more accurately. Suchita and Lalit \cite{bac5}
	proposed unsupervised neural network learning technique for classification
	of brain MRI images. The MRI brain images are first preprocessed which
	include noise filtering, edge detection, then the tumor is extracted using
	segmentation. The texture features are extracted using Gray-Level
	Co-occurrence Matrix(GLCM) and then Self-Organizing Maps (SOM) are used to
	classify the brain as normal or abnormal brain, that is, whether it contain
	tumor or not. Rajeshwari and Sharmila \cite{bac6} proposed preprocessing
	techniques which are used to improve the quality of MRI image before using
	it into an application. The average, median and wiener filters are used for
	noise removal and interpolation based Discrete Wavelet Transform (DWT)
	technique is used for resolution enhancement. The Peak Signal to Noise Ratio
	(PSNR) is used for evaluation of these techniques.

	George and Karnan \cite{George2012MRIBI} proposed MRI image enhancement
	technique based on Histogram Equalization and Center Weighted Median (CWM)
	filter as they are used to enhance the MRI image more effectively. Daljit
	Singh et \cite{Funmilola2015ClassificationOA} proposed a hybrid technique
	for automatic classification of MRI images by first extracting the features
	using Principal Component Analysis (PCA) and Gray-Level Co-occurrence
	Matrix(GLCM) and then extracted features are fed as an input to Support
	Vector Machine(SVM) classifier which classifies the brain image as normal or
	abnormal. Gadpayleand and Mahajani \cite{9074375} proposed brain tumour
	detection and classification system. The tumor is extracted using
	segmentation and then texture features are extracted using GLCM and finally
	the BPNN and KNN classifiers are used to classify the MRI brain image into
	normal or abnormal brain. The accuracy is 70\% using KNN classifier and
	72.5\% by using BPNN classifier. Shasidhar et al. in \cite{article} proposed
	modified Fuzzy C-Means (FCM) algorithm for MR brain tumor detection. The
	texture features are extracted from brain MR image and then modified FCM
	algorithm is used for brain tumor detection. The average speed-ups of as
	much as 80 times a traditional FCM algorithm is obtained using the modified
	FCM algorithm. The modified FCM algorithm is a fast alternative to the
	traditional FCM technique.  Rajesh and Malar \cite{inproceedings} proposed
	brain MR image classification based on Rough set theory and feed-forward
	neural network classifier. The features are extracted from MRI images using
	Rough set theory. The selected features are fed as input to Feed Forward
	Neural Network classifier which differentiate between normal and abnormal
	brain and the accuracy of about 90\% is obtained.

	Ramteke and Monali \cite{article2} proposed automatic classification of
	brain MR images in two classes Normal and Abnormal based on image features
	and automatic abnormality detection. The Statistical texture feature set is
	obtained from normal and abnormal images and then KNN classifier is used for
	classifying image. The KNN obtain 80\% classification rate.  Xuan and Liao
	\cite{4297123} proposed statistical structure analysis based tumor
	segmentation technique. The intensity-based, symmetry-based and texture-
	based features are extracted from MR image. Then, classification technique
	using AdaBoost is used to classify the MR image into normal tissues and
	abnormal images. The average accuracy of about 96.82\% is achieved. Othman
	et al. in \cite{5730335} proposed Probabilistic neural network technique for
	brain tumor classification. Firstly, the features are extracted using the
	principal component analysis (PCA) and the classification is performed using
	Probabilistic Neural Network (PNN).  Ibrahim et al. in \cite{inproceedings2}
	proposed Neural Network technique for the classification of the magnetic
	resonance human brain images. The features are extracted using principal
	Component Analysis (PCA) and then Back- Propagation Neural Network is used
	as a classifier to classify MRI brain images as normal or abnormal. The
	classification accuracy of about 96.33\% is obtained. Jafari and Shafaghi
	\cite{article3} proposed a hybrid approach for brain tumor detection in MR
	images based on Support Vector Machines(SVM). The texture and intensity
	features are used. The accuracy of about 83.22\% is achieved and is more
	robust.

	Thus from extensive literature survey we found that most of the current
	brain tumor detection system uses texture, symmetry and intensity as
	features. Texture features are important property of brain as texture
	perception has a very important aspect in the human visual system of
	recognition and interpretation \cite{article4}. Here, we propose extracting
	texture features like energy, contrast, correlation, Homogenity
	\cite{6524466}. Gray Level Co-occurrence Matrix is used for extraction of
	texture features.

	Further we propose the use of ML algorithms to overcome the drawbacks of
	traditional classifiers. We investigate and compare the performance of two
	machine learning algorithms namely MLP and Naive Bayes in this work. Since
	these ML algorithms are found to perform well in most of the pattern
	classification tasks. Neural networks are useful as they can learn complex
	mappings between input and output. They are capable of solving much more
	complicated classification tasks. However, when certain rules cannot be
	modeled exactly, the concept of probability is used, which is the basis for
	Naive Bayes classification.
